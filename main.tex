\documentclass[twocolumn]{el-author}

%\usepackage[...]{...}      This has been commented out as we are not using any additional packages here.  On the whole, they should be unnecessary.
\newcommand{\hH}{\hat{H}}
\newcommand{\D}{^\dagger}
\newcommand{\ua}{\uparrow}
\newcommand{\nc}{\newcommand}
\nc{\da}{\downarrow} \nc{\hc}{\hat{c}} \nc{\hS}{\hat{S}}
\nc{\bra}{\langle} \nc{\ket}{\rangle} \nc{\eq}{equation (\ref}
\nc{\h}{\hat} \nc{\hT}{\h{T}}\nc{\be}{\begin{eqnarray}}
\nc{\ee}{\end{eqnarray}}\nc{\rd}{\textrm{d}}\nc{\e}{eqnarray}\nc{\hR}{\hat{R}}\nc{\Tr}{\mathrm{Tr}}
\nc{\tS}{\tilde{S}}\nc{\tr}{\mathrm{tr}}\nc{\8}{\infty}\nc{\lgs}{\bra\ua,\phi|}\nc{\rgs}{|\ua,\phi\ket}
\nc{\hU}{\hat{U}}\nc{\lfs}{\bra\phi|}\nc{\rfs}{|\phi\ket}\nc{\hZ}{\hat{Z}}\nc{\hd}{\hat{d}}\nc{\mD}{\mathcal{D}}
\nc{\bd}{\bar{d}}\nc{\bc}{\bar{c}}\nc{\mc}{\mathcal}\nc{\ea}{eqnarray}\nc{\mG}{\mathcal{G}}\nc{\bce}{\begin{center}}
\nc{\ece}{\end{center}}
\date{25th December 2017}

\usepackage{color}
\usepackage{ascmac}
\usepackage{amsmath}
\usepackage{graphicx}
\graphicspath{{./Figure/}}

\begin{document}

\title{Proof-of-concept for an all-optical AND gate with photonic crystal quantum dot semiconductor optical amplifiers}

\author{T. Matsumoto, G. Hosoya and H. Yashima}

\abstract{%All-optical logic gates(AOLGs) are composed of all-optical devices which can operate as well as transistor under high-speed bit rates and essential for all optical communication network. Althought many researches on AOLGs have been demonstrated, discussions about the circuit integration are few. 
An all-optical AND gate that uses photonic-crystal (PC) quantum-dot semiconductor optical amplifiers (QDSOAs) is designed and simulated herein. The input-output characteristics of the gate are numerically analyzed using rate equation modeling to prove that the gate can operate at 160 Gb/s. The proposed gate is compared with the design that uses an optical gate to evaluate its effectiveness with regard to signal output, device size, and power consumption.}

\maketitle

\section{Introduction}
Owing to progressing technology, high-speed and large-capacity data transmission is required to cope with increasing network traffic  {\cite{qdsoa_ethernet}}. In current optical networks, an optical signal travels to its destination through optical fibers, and the signal is affected by noise along the way. Thus, re-amplification, re-timing, and re-shaping of the signal are essential for a reliable optical network operation. Currently, electronic components are used to implement these operations. However, these components have low processing speed and consume high energy during optical-to-electrical signal conversion {\cite{3r_regeneration}}. Signal processing equipment that uses all-optical components can avoid these limitations, and researchers are actively pursuing all-optical signal processing components.
QDSOAs are optical devices that can be useful in many applications owing to its relatively fast gain recovery time and nonlinear optics {\cite{qdsoa_aosp}}. The confinement of electrons and electron holes in QDs enables fast gain recovery and gain saturation in the SOA leads to the nonlinear optics.
In contrast, PC is a dielectric material with a periodic pattern of two different refractive indexes, therefore, light waves of a single bandwidth cannot propagate through PC. PC waveguides (PCWs) exploit this characteristic with a line-shaped recession in a PC slab. This linear recession confines the light waves in the directions perpendicular and parallel to the direction of propagation. Moreover, the dispersion relation between the PC refractive indices can decelerate the wave velocity, which can be useful in cases wherein a PCW is combined with an SOA or QDSOA. \par
Similar to electronic logic gates, all-optical logic gates (AOLGs) can manipulate binary inputs. Although many AOLGs have been developed using a variety of optical materials, the length of these AOLGs tends to be larger than that of electrical logic gates since all-optical components must be long enough to accommodate the reflections of the signal. For instance, a typical QDSOA is approximately 2-mm long {\cite{qdsoa_nssm}}, whereas the equivalent transistor used in commercially available electronic devices is approximately 14-nm long {\cite{current_cpu}}.
Few studies have focused on the length of the components in optical logic gates, although this is an important concerning the development of devices with high-density integration and minimal energy consumption.
Therefore, an all-optical AND gate that uses PC-QDSOA components is proposed herein. AND gates play an important role for multiplexer and demultiplexer circuits that are commonly used in communication systems, computer memory, and arithmetic logic units. To demonstrate the effectiveness of the proposed gate, we numerically analyzed its input-output characteristics using rate-equation models. To our knowledge, no research has been conducted on the numerical analysis of a PC-QDSOA all-optical AND gate yes. Our results indicate that an all-optical AND gate with the proposed PC-QDSOA waveguide could feasibly operate at 160 Gb/s and can achieve a maximum of approximately 9-dB extinction ratio (ER). In addition, relatively low-current injection is required when using the proposed all-optical AND gate.

\section{PC-QDSOA model}
Fig. {\ref{fig:pcqdsoa}} schematizes a PC-QDSOA waveguide. The PC-QDSOAs used in this study comprise GaAs, and ${\rm{In_{0.15}Ga_{0.85}As}}$ and include InAs at the active region to enhance the optical power by ${\rm{1.3 \mu m}}$ {\cite{theory_of_qdsoa}}. Population inversion in the active regions of the amplifiers is achieved by passing current through laterally doped ${\rm{p^{+}-p-n^{+}}}$ structures. The W1 PCW of the PC-QDSOA allows slow light to be achieved with zero group-velocity and zero third-order dispersions at a specific bandwidth. The W1 PCW is fabricated by creating a line-shaped recession in a PC slab, and the slab is fabricated by creating periodic round vacancies on n- and p- clad regions {\cite{how_to_create_pcw}}. 

\begin{figure}[htbp]
\begin{center}
  \includegraphics[width=70mm]{pcqdsoa.eps}
  \caption{Schematic of the PC-QDSOA waveguide}
  \label{fig:pcqdsoa}
\end{center}
\end{figure}
The operation of the PC-QDSOAs can be studied theoretically using the rate-equation model {\cite{pcqdsoa}}. InAs QDs comprise a ground state (GS), an excited state (ES), and an upper state (US). Quantum wells (QWs) are the common carrier reservoirs in QDs. QDs exhibit homogeneous broadened sizes and shapes, therefore, it is assumed that these states have 2M+1 variables. Thus, the rate equation for the PC-QDSOA is as follows:

\begingroup\makeatletter\def\f@size{7}\check@mathfonts
\begin{eqnarray}
  \label{eq:pc_qd_qw}
  \frac{ {\partial}f^{c(v)}_{w} }{ {\partial}t } & = & 
	\frac{\eta_{inj}I}{qN^{c(v)}_{W}} - 
	\frac{ \sqrt{ f^{c}_{w} f^{v}_{w} } }{ \hat{\tau}^{c(v)}_{wr} } + 
	\frac{ D^{c(v)}_{u} }{ D^{c(v)}_{w} } \sum_{j=1}^{2M+1} G^{c(v)}_{j} \nonumber\\ &&
	\times
	\left(
		\frac{ f^{c(v)}_{u,j} }{ \hat{\tau}^{c(v)}_{uw,j} }	\left(1 - f^{c(v)}_{w} \right) -
		\frac{ f^{c(v)}_{w}  }{ \hat{\tau}^{c(v)}_{wu}  } 	\left(1 - f^{c(v)}_{u,j}\right)
	\right)
  \\
  \label{eq:pc_qd_us}
  \frac{ {\partial}f^{c(v)}_{u,j} }{ {\partial}t } & = & 
	\frac{ f^{c(v)}_{w}  }{ \hat{\tau}^{c(v)}_{wu}  }	\left(1 - f^{c(v)}_{u,j}\right) -
	\frac{ f^{c(v)}_{u,j} }{ \hat{\tau}^{c(v)}_{uw,j} }	\left(1 - f^{c(v)}_{w} \right) + \frac{ D^{c(v)}_{e} }{ D^{c(v)}_{u} } \nonumber\\ &&
	\times \left(
		\frac{ f^{c(v)}_{e,j} }{ \hat{\tau}^{c(v)}_{eu} } \left(1 - f^{c(v)}_{u,j}\right) -
		\frac{ f^{c(v)}_{u,j} }{ \hat{\tau}^{c(v)}_{ue} } \left(1 - f^{c(v)}_{e,j}\right)
	\right)  \nonumber\\ &&
	+ \frac{ D^{c(v)}_{g} }{ D^{c(v)}_{u} } \left(
		\frac{ f^{c(v)}_{g,j} }{ \hat{\tau}^{c(v)}_{gu} } \left(1 - f^{c(v)}_{u,j}\right) -
		\frac{ f^{c(v)}_{u,j} }{ \hat{\tau}^{c(v)}_{ug} } \left(1 - f^{c(v)}_{g,j}\right)
	\right)  \nonumber\\ &&
	- \frac{ \sqrt{ f^{c}_{u,j} f^{v}_{u,j} } }{ \hat{\tau}^{c(v)}_{ur} }
  \\
  \label{eq:pc_qd_es}
  \frac{ {\partial}f^{c(v)}_{e,j} }{ {\partial}t } & = & 
	\frac{ f^{c(v)}_{u,j} }{ \hat{\tau}^{c(v)}_{ue} }	\left(1 - f^{c(v)}_{e,j}\right) -
	\frac{ f^{c(v)}_{e,j} }{ \hat{\tau}^{c(v)}_{eu} }	\left(1 - f^{c(v)}_{u,j}\right) \nonumber\\ &&
	+ \frac{ D^{c(v)}_{g} }{ D^{c(v)}_{e} } \left(
		\frac{ f^{c(v)}_{g,j} }{ \hat{\tau}^{c(v)}_{ge} } \left(1 - f^{c(v)}_{e,j}\right) -
		\frac{ f^{c(v)}_{e,j} }{ \hat{\tau}^{c(v)}_{eg} } \left(1 - f^{c(v)}_{g,j}\right)
	\right)  \nonumber\\ &&
	- \frac{ 1 }{ N^{c(v)}_{E,j} } \sum_{k} \frac
		{\Gamma_{k}P_{k,in}g^{e}_{j,k}\left[e^{\left(\left[g_{mod}(t,\lambda_{k}) - \alpha(\lambda_{k})\right]L_{ca}\right)} - 1\right]}
		{\hbar \omega_{k}\left[g_{mod}(t,\lambda_{k}) - \alpha(\lambda_{k})\right]} \nonumber\\ &&
	\times
	\left( f^{c}_{e,j} + f^{v}_{e,j} - 1 \right)
  \\
  \label{eq:pc_qd_gs}
  \frac{ {\partial}f^{c(v)}_{g,j} }{ {\partial}t } & = & 
	\frac{ f^{c(v)}_{u,j} }{ \hat{\tau}^{c(v)}_{ug} }	\left(1 - f^{c(v)}_{g,j}\right) -
	\frac{ f^{c(v)}_{g,j} }{ \hat{\tau}^{c(v)}_{gu} }	\left(1 - f^{c(v)}_{u,j}\right) \nonumber\\ &&
	\frac{ f^{c(v)}_{e,j} }{ \hat{\tau}^{c(v)}_{eg} } \left(1 - f^{c(v)}_{g,j}\right) -
	\frac{ f^{c(v)}_{g,j} }{ \hat{\tau}^{c(v)}_{ge} } \left(1 - f^{c(v)}_{e,j}\right) \nonumber\\ &&
	- \frac{ \sqrt{ f^{c}_{g,j} f^{v}_{g,j} } }{ \hat{\tau}_{dr} } - \frac{1}{N^{c(v)}_{G,j}} \nonumber\\ &&
	\times \left(\sum_{k} 
	\frac
		{\Gamma_{k}P_{k,in}g^{g}_{j,k}\left[e^{\left(\left[g_{mod}(t,\lambda_{k}) - \alpha(\lambda_{k})\right]L_{ca}\right)} - 1\right]}
		{\hbar \omega_{k}\left[g_{mod}(t,\lambda_{k}) - \alpha(\lambda_{k})\right]} + \right. \nonumber\\ && 
	\left.
	\frac
		{\Gamma_{p}P_{p}g^{e}_{j,p}\left[e^{\left(\left[g_{mod}(t,\lambda_{p}) - \alpha(\lambda_{p})\right]L_{ca}\right)} - 1\right]}
		{\hbar \omega_{p}\left[g_{mod}(t,\lambda_{p}) - \alpha(\lambda_{p})\right]}
	\right) \nonumber\\ &&
	\times
	\left( f^{c}_{g,j} + f^{v}_{g,j} - 1 \right)
\end{eqnarray}
\endgroup
The term $f^{c(v)}_{w}$ represents the carrier occupancy of the QW. Likewise, $f^{c(v)}_{u,j},f^{c(v)}_{e,j},f^{c(v)}_{g,j}$ represent the carrier occupancy of the $j$th group of US, ES, and GS respectively. The term $\Gamma_{k}$ is the optical confinement factor at wavelength $\lambda_{k}$. The terms $P_{k,in}$ and $P_{p}$ represent the power of the $k$th photon mode and probe signal, respectively. The term $\alpha(\lambda)$ represents the loss coefficient at the wavelength $\lambda$, which is the sum of the scattering and absorption losses. The term $g_{mod}(t,\lambda_{k})$ is the effect on linear modal gain caused by slow light, and it can be expressed as the product of the slowdown factor, optical-confinement factor, and linear material gain of the active region.
% ���ڂ���
The term $g^{g(e)}_{j,k}$ is the linear optical gain that the GS (ES) of the $j$-th QD group provides to the $k$-th photon mode. More details about the PC-QDSOA structure are provided in reference {\cite{pcqdsoa}}.

\section{AND gate model}
Fig. {\ref{fig:and_gate}} schematizes the PC-QDSOA all-optical AND gate. Following the labels in Fig. {\ref{fig:and_gate}}, the operation of AND gate is described as follows: a modulated data signal A (at wavelength $\lambda_{A}$) and a clock signal (at wavelength $\lambda_{C}$) are input to the PC-QDSOA1. Signal A will induce low gain amplification on the clock signal via cross-gain modulation (XGM) in PC-QDSOA1, therefore, the logic output is always NOT A. In the same way as this operation, the modulated data signal NOT A (at wavelength $\lambda_{C}$) and a modulated data signal B (at wavelength $\lambda_{B}$, which can be equal to the wavelength of signal A) are input to PC-QDSOA2, and the logic output is then A AND B.

\begin{figure}[htbp]
\begin{center}
  \includegraphics[width=80mm]{and_gate.eps}
  \caption{Schematic of the PC-QDSOA all-optical AND gate}
  \label{fig:and_gate}
\end{center}
\end{figure}

\section{Results}
The operation of the proposed gate design is numerically analyzed using MATLAB 2016b and Optisystem 14.0.0. The physical parameters used for solving the rate equation are provided in reference {\cite{pcqdsoa}}. Table. {\ref{tb:param}} lists the fixed parameters used for the numerical analysis in this paper. The pulses were Gaussian shaped. \par
\begin{table}[htbp]
  \begin{center}
  \caption{Fixed parameters used in numerical analysis}
  \begin{tabular}{|c|c|c|} \hline
    Parameter & Value & Unit \\ \hline
    Maximum power of Input A & 10 & mW \\ \hline
    Maximum power of Input B & 1 & $\mu W$ \\ \hline
    Maximum power of Clock & 100 & $\mu W$ \\ \hline
    Wavelength of Input A & 1307 & nm \\ \hline
    Wavelength of Input B & 1307 & nm \\ \hline
    Wavelength of Clock & 1310 & nm \\ \hline
    Full width at half maximum of pulse & 1.2 & ps \\ \hline
    Transmission speed & 160 & Gb/s \\ \hline
  \end{tabular}
  \label{tb:param}
  \end{center}
\end{table}
To evaluate the proposed gate, the eye diagram, ER, and Q-factor are used as metrics. The ER represents can be represented as $ER[dB] = 10\log_{10} \left(P^{1}_{min} / P^{0}_{max} \right)$. $P^{1}_{min}$ represents minimum power of the binary signal ``1'' and $P^{0}_{max}$ represents the maximum power of the ``0'' signal. The Q-factor can be represented as $Q = (S_{1}-S_{0})/(\sigma_{1}+\sigma_{0})${\cite{q_factor}} where $S_{1}$, and $S_{0}$ are the average powers of signals ``1'' and ``0'' and $\sigma_{1}$, and $\sigma_{0}$ are the standard deviations of those signals. Herein, the large numbers of these metrics represent the proposed gate is more appropriate for an all-optical AND gate.\par
Fig.{\ref{fig:output_signal}} shows the simulation results for the input-output characteristics with 6-mA current injection. Fig. {\ref{fig:eye_dia}} shows an eye diagram of the output signal. The ER and Q-factor for the output signal are 8.58 dB and 7.41, respectively. These results show that the proposed gate design can operate as an AND gate at 160 Gb/s {\cite{extinction_ratio}}.
\begin{figure}[htbp]
\begin{center}
  \includegraphics[width=75mm]{in-out.eps}
  \caption{Input-output characteristics for PC-QDSOA all-optical AND gate when current injection is 6 mA}
  \label{fig:output_signal}
\end{center}
\end{figure}
\begin{figure}[htbp]
\begin{center}
  \includegraphics[width=75mm]{eyedia_6mA.eps}
  \caption{Eye diagram of the output signal with 6-mA current injection}
  \label{fig:eye_dia}
\end{center}
\end{figure}
Since the gain recovery time varies with the level of current injection, we also investigated the effect of varying current injection on the power output from the proposed AND gate. Figs. {\ref{fig:pcqdsoa_different_pump_current_ERs}} and {\ref{fig:pcqdsoa_different_pump_current_Qs}}, respectively, show the ERs and Q-factors of the output signals with different injected currents. The ERs and Q-factors improve with increasing current injection because pattern effects decrease. When current injection exceeds 9 mA, the ERs change slightly because the maximum gain-recovery time is limited by carrier-relaxation and capture times. 

\begin{figure}[htbp]
\begin{center}
  \includegraphics[width=80mm]{pcqdsoa_ERs.eps}
  \caption{ERs with varying current injection}
  \label{fig:pcqdsoa_different_pump_current_ERs}
\end{center}
\end{figure}

\begin{figure}[htbp]
\begin{center}
  \includegraphics[width=80mm]{pcqdsoa_Qs.eps}
  \caption{Q-factor under different current injection}
  \label{fig:pcqdsoa_different_pump_current_Qs}
\end{center}
\end{figure}
\newpage
To quantify the effectiveness of the proposed gate, the proposed AND gate is compared with QDSOA-assisted gate without PC components. This design follows the same schematic diagram as the proposed design. Physical parameters and the methods used to analyze this QDSOA design are provided in reference {\cite{qdsoa_nssm}}. Figs. {\ref{fig:comp_ERs}} and {\ref{fig:comp_Qs}}, respectively, plot ERs and Q-factors vs. current injection for QDSOA-assisted AND gate and the proposed design. To obtain ER of approximately 7.5 dB, the QDSOA-assisted AND gate requires 3000 mA, whereas the proposed design requires only 5-mA injected current. Likewise, Fig. {\ref{fig:comp_Qs}} shows that to obtain a Q-factor of approximately 5, the QDSOA-assisted AND gate requires 2800-mA injected current, whereas the proposed design requires only 4 mA. Moreover, the QDSOA is 2-mm long, whereas the proposed PC-QDSOA is 125-${\rm{\mu m}}$ long. This comparison demonstrates that the AND gate reduces energy consumption and device volume compared with all-optical AND gates reported in the literature.

\begin{figure}[htbp]
\begin{center}
  \includegraphics[width=80mm]{qdsoa_vs_pcqdsoa_ERs.eps}
  \caption{ERs under different current injection}
  \label{fig:comp_ERs}
\end{center}
\end{figure}

\begin{figure}[htbp]
\begin{center}
  \includegraphics[width=80mm]{qdsoa_vs_pcqdsoa_Qs.eps}
  \caption{Q-factor under different current injection}
  \label{fig:comp_Qs}
\end{center}
\end{figure}

\section{Conclusion}
We designed a PC-QDSOA all-optical AND gate that can operate at 160 Gb/s. The measures of eye diagram, ER, and Q-factor quantify the design's performance. Numerical analyses show that the proposed gate can operate as an AND gate at 160 Gb/s when current injection exceeds 6 mA. This performance can be improved by increasing current injection, as this would decrease pattern effects.

\vskip3pt

\noindent T. Matsumoto, G. Hosoya and H. Yashima (\textit{Tokyo University of Science, 6-3-1, Shinjuku, Katsushika-ku, Tokyo, 125-8585, Japan})
\vskip3pt

\noindent E-mail: 4416630@ed.tus.ac.jp

\begin{thebibliography}{9}
\bibitem{qdsoa_ethernet}
D. Bimberg, M. Laemmlin, C. Meuer, G. Fiol, M. Kuntz, A. Schliwa, N. N. Ledentsov and A. R. Kovsh, ``Quantum Dot Amplifiers for 100 Gbit Ethernet, " {\it ICTON}, pp. 1924--1930, Jun. 2006.

\bibitem{3r_regeneration}
B. Sartorius, ``3R regeneration for all-optical networks, " {\it ICTON}, pp. 333--337, Aug. 2002.

\bibitem{qdsoa_aosp}
I. Kang, C. Dorrer, L. Zhang, M. Dinu, M. Rasras, L. L. Buhl, S. Cabot, A. Bhardwaj, X. Liu, M. A. Cappuzzo, L. Gomez, A. Wong-Foy, Y. F. Chen, N. K. Dutta, S. S. Patel, D. T. Neilson, C.R. Giles, A. Piccirilli and J. Jaques, ``Characterization of the dynamical processes in all-optical signal processing using semiconductor optical amplifiers, " {\it IEEE J. Sel. Topics Quantum Electron.}, vol. 14, no.3, pp. 758--769, May/Jun. 2008.

\bibitem{qdsoa_nssm} 
K. Abedi and H. Taleb, ``Phase Recovery Acceleration in Quantum-Dot Semiconductor Optical Amplifiers, " {\it J. Lightw. Technol.}, vol. 32, no.12, pp. 237--241, Jun. 2012.

\bibitem{current_cpu}
(2017, Dec. 8). {\it Intel Core i7-5557U specifications} [Online]. Available: http://www.cpu-world.com/CPUs/Core\_i7/Intel-Core\%20i7-5557U\%20Mobile\%20processor.html

\bibitem{theory_of_qdsoa}
M. Sugawara, H. Ebe, N. Hatori, M. Ishida, Y. Arakawa, T. Akiyama, K. Otsubo and Y. Nakata, ``Theory of optical signal amplification and processing by quantum-dot semiconductor optical amplifiers, '' {\it PhysRevB}, vol. 69, no. 23, Jun. 2004.

\bibitem{how_to_create_pcw}
O. Khayam and H. Benisty, ``General recipe for flatbands in photonic crystal waveguides, '' {\it Opt. Exp.}, vol. 17, no. 17, pp. 14634--14648, Aug. 2009.

\bibitem{pcqdsoa} 
H. Taleb and K. Abedi, ``Optical Gain, Phase, and Refractive Index Dynamics in Photonic Crystal Quantum-Dot Semiconductor Optical Amplifiers, " {\it IEEE J. Quantum Electron.}, vol. 50, no. 8, pp. 605--612, Aug. 2014.

\bibitem{q_factor} 
P. Agrawal, ``Fiber-Optic Communication Systems, third ed., " {\it Wiley, John \& Sons.}, May. 2002.

\bibitem{extinction_ratio} 
D. Gayen, A. Bhattachryya, T. Chattopadhyay and J. Roy, ``Ultrafast All-Optical Half Adder Using Quantum-Dot Semiconductor Optical Amplifier-Based Mach-Zehnder Interferometer, " {\it J. Lightw. Technol.}, vol. 30, no. 21, pp. 3387--3393, Sep. 2012.

\end{thebibliography}

\end{document}
