\documentclass[twocolumn]{el-author}

%\usepackage[...]{...}      This has been commented out as we are not using any additional packages here.  On the whole, they should be unnecessary.
\newcommand{\hH}{\hat{H}}
\newcommand{\D}{^\dagger}
\newcommand{\ua}{\uparrow}
\newcommand{\nc}{\newcommand}
\nc{\da}{\downarrow} \nc{\hc}{\hat{c}} \nc{\hS}{\hat{S}}
\nc{\bra}{\langle} \nc{\ket}{\rangle} \nc{\eq}{equation (\ref}
\nc{\h}{\hat} \nc{\hT}{\h{T}}\nc{\be}{\begin{eqnarray}}
\nc{\ee}{\end{eqnarray}}\nc{\rd}{\textrm{d}}\nc{\e}{eqnarray}\nc{\hR}{\hat{R}}\nc{\Tr}{\mathrm{Tr}}
\nc{\tS}{\tilde{S}}\nc{\tr}{\mathrm{tr}}\nc{\8}{\infty}\nc{\lgs}{\bra\ua,\phi|}\nc{\rgs}{|\ua,\phi\ket}
\nc{\hU}{\hat{U}}\nc{\lfs}{\bra\phi|}\nc{\rfs}{|\phi\ket}\nc{\hZ}{\hat{Z}}\nc{\hd}{\hat{d}}\nc{\mD}{\mathcal{D}}
\nc{\bd}{\bar{d}}\nc{\bc}{\bar{c}}\nc{\mc}{\mathcal}\nc{\ea}{eqnarray}\nc{\mG}{\mathcal{G}}\nc{\bce}{\begin{center}}
\nc{\ece}{\end{center}}
\date{30th January 2018}

\usepackage{color}
\usepackage{ascmac}
\usepackage{amsmath}
\usepackage{graphicx}

\begin{document}

\title{Performance of all-optical AND gate using photonic-crystal QDSOA at 160 Gb/s}

\author{T. Matsumoto, K. Komatsu, G. Hosoya and H. Yashima}

\abstract{An all-optical AND gate using photonic-crystal quantum dot semiconductor optical amplifiers is designed and its performance is evaluated. The input-output characteristics of the gate are simulated using rate equation model and it is found that the gate can achieve a maximum of approximately 9-dB extinction ratio at 160 Gb/s. The proposed gate is compared with quantum dot semiconductor optical amplifiers AND gate to evaluate its effectiveness with regard to signal quality, device size, and power consumption.}

\maketitle

\section{Introduction}
Owing to progressing technology, high-speed and large-capacity data transmission is required to cope with increasing network traffic. In current optical networks, re-amplification, re-timing, and re-shaping of the signal are essential for a reliable optical network operation. Currently, electronic components are used to implement these operations. However, these components have low processing speed and consume high energy during optical-to-electrical signal conversion {\cite{3r_regeneration}}. Signal processing equipment that uses all-optical components can avoid these limitations, and researchers are actively pursuing all-optical signal processing components.
Quantum dot semiconductor optical amplifiers (QDSOAs) is an optical device that can be useful in many applications owing to its relatively fast gain recovery time and nonlinear optics. The confinement of electrons and electron holes in QDs enables fast gain recovery, and gain saturation in the SOA leads to the nonlinear optics.
In contrast, photonic crystal (PC) is a dielectric material with a periodic pattern of two different refractive indexes, therefore, light waves of a single bandwidth cannot propagate through PC. PC waveguide (PCW) can decelerate the wave velocity, which can be useful in cases wherein a PCW is combined with an SOA or QDSOA. \par
Similar to electronic logic gates, all-optical logic gates (AOLGs) can manipulate binary inputs. Although many AOLGs have been developed using a variety of optical materials, the length of these AOLGs tends to be larger than that of electrical logic gates since all-optical components must be long enough to operate the signal. For instance, a typical QDSOA is approximately 2-mm long {\cite{qdsoa_nssm}}, whereas the equivalent transistor used in commercially available electronic devices is approximately 14-nm long {\cite{current_cpu}}. This is an important concerning of the development of devices with high-density integration and minimal energy consumption. Meanwhile, AND gates is one of the most commonly used logic gate. To our knowledge, no research has been conducted on a PC-QDSOA all-optical AND gate yet. \par
In this paper, an all-optical AND gate using PC-QDSOA components is proposed. To demonstrate the effectiveness of the proposed gate, we simulate its input-output characteristics using rate-equation model. Our results indicate that an all-optical AND gate with the PC-QDSOA waveguide feasibly operates at 160 Gb/s and can achieve a maximum of approximately 9-dB extinction ratio (ER) using the parameters shown in the results section. In addition, it is shown that substantially low-current injection is required when using the proposed all-optical AND gate.

\section{PC-QDSOA model}
The PC-QDSOAs used in this study comprise GaAs, ${\rm{In_{0.15}Ga_{0.85}As}}$, and InAs at the active region to enhance the optical power whose wavelength is nearby approximately ${\rm{1.3 \mu m}}$ {\cite{theory_of_qdsoa}}. Population inversion in the active regions of the amplifiers is achieved by passing current through doped ${\rm{p^{+}-p-n^{+}}}$ structures. The W1 PCW of the PC-QDSOA allows slow light to be achieved with zero group-velocity dispersion and zero third-order dispersion at a specific bandwidth. More details about the PC-QDSOA structure are provided in reference {\cite{pcqdsoa}}.

\section{AND gate model}
Fig. {\ref{fig:and_gate}} schematizes the PC-QDSOA all-optical AND gate. Following the labels in Fig. {\ref{fig:and_gate}}, the operation of AND gate is described as follows: a modulated data signal A (at wavelength $\lambda_{A}$) and a clock signal (at wavelength $\lambda_{C}$) are injected to the PC-QDSOA1. Signal A with high intensity causes carrier depletion which leads to gain saturation. It causes intensity reduction of clock signal in PC-QDSOA1, therefore, the logic output is always NOT A. In the same way as this operation, the modulated data signal NOT A (at wavelength $\lambda_{C}$) and a modulated data signal B (at wavelength $\lambda_{B}$, which can be equal to the wavelength of signal A) are injected to PC-QDSOA2, and the logic output is then A AND B. Though, the structure of some of other all-optical logic gates using QDSOA is based on Mach-Zehnder interferometer (MZI) {\cite{mzi_gate}} with exploiting cross phase modulation, the proposed gate is less complex with exploiting cross gain modulation. MZI based all-optical gates require the devices to have same characteristics, but owing to the schema, the proposed gate does not necessarily require the PC-QDSOAs to have same characteristics.

\begin{figure}[htbp]
\begin{center}
  \includegraphics[width=100mm,bb=0 0 980 300]{and_gate.pdf}
  \caption{Schematic of the PC-QDSOA all-optical AND gate}
  \label{fig:and_gate}
\end{center}
\end{figure}

\section{Results}
The operation of the proposed gate is simulated using MATLAB 2016b and Optisystem 14.0.0. The rate-equation used for theoretically studying the operation of the PC-QDSOAs and the physical parameters used for the equation are provided in reference {\cite{pcqdsoa}}. The fixed parameters used for the following simulations are as follows: The maximum power of Input A, Input B, and Clock are 10 mW, 1 $\mu W$, and 100 $\mu W$, respectively. The wavelength of Input A, Input B, and Clock are 1307 nm, 1307 nm, and 1310 nm, respectively. The pulses are Gaussian shaped with 1.2-ps full width at half maximum (FWHM), and the bitrate is 160Gb/s. \par

\begin{figure}[htbp]
\begin{center}
  \includegraphics[width=100mm,bb=0 0 960 520]{in-out.pdf}
  \caption{Input-output characteristics for PC-QDSOA all-optical AND gate when current injection is 6 mA}
  \label{fig:output_signal}
  \includegraphics[width=80mm,bb=0 0 720 580]{eyedia_6mA.pdf}
  \caption{Eye diagram of the output signal with 6-mA current injection}
  \label{fig:eye_dia}
\end{center}
\end{figure}

\begin{figure}[htbp]
\begin{center}
  \includegraphics[width=80mm,bb=0 0 850 500]{pcqdsoa_ERs.pdf}
  \caption{ERs with varying current injection}
  \label{fig:pcqdsoa_different_pump_current_ERs}
  \includegraphics[width=80mm,bb=0 0 850 500]{pcqdsoa_Qs.pdf}
  \caption{Q-factors with varying current injection}
  \label{fig:pcqdsoa_different_pump_current_Qs}
  \includegraphics[width=100mm,bb=0 0 950 560]{qdsoa_vs_pcqdsoa_ERs.pdf}
  \caption{ERs of QDSOA and PC-QDSOA all-optical AND gate with varying current injection}
  \label{fig:comp_ERs}
  \includegraphics[width=100mm,bb=0 0 950 560]{qdsoa_vs_pcqdsoa_Qs.pdf}
  \caption{Q-factors of QDSOA and PC-QDSOA all-optical AND gate with varying current injection}
  \label{fig:comp_Qs}
\end{center}
\end{figure}
\newpage
To evaluate the proposed gate, the eye diagram, ER, and quality factor (Q-factor) are used as metrics. The ER can be represented as $ER[dB] = 10\log_{10} \left(P^{1}_{min} / P^{0}_{max} \right)$. $P^{1}_{min}$ represents minimum power of the binary signal ``1'' and $P^{0}_{max}$ represents the maximum power of the binary signal ``0''. The Q-factor can be represented as $Q = (S_{1}-S_{0})/(\sigma_{1}+\sigma_{0}) $ where $S_{1}$, and $S_{0}$ are the average powers of binary signals ``1'' and ``0'', and $\sigma_{1}$ and $\sigma_{0}$ are the standard deviations of those signals. \par
Figs. {\ref{fig:output_signal}} and {\ref{fig:eye_dia}} show the simulation results for the input-output characteristics, and eye diagram of the output signal with 6-mA current injection, respectively. The ER and Q-factor of the output signal are 8.58 dB and 7.41, respectively. These results show that the proposed gate can operate as an AND gate at 160 Gb/s. We also investigate the effect of varying current injection on the output signal. Figs. {\ref{fig:pcqdsoa_different_pump_current_ERs}} and {\ref{fig:pcqdsoa_different_pump_current_Qs}} show the ERs and Q-factors of the output signals with different current injections, respectively. The ERs and Q-factors improve with increasing current injection because pattern effects decrease. When current injection $I \geq$ 9 mA, the ERs change slightly because the maximum gain-recovery time is limited by carrier-relaxation and capture times. \par
To quantify the effectiveness of the proposed gate, the performance is compared with those of QDSOA AND gate. Schematic diagram of the gate is same as the proposed design. Physical parameters and the equations used for simulation of the QDSOA AND gate are provided in reference {\cite{qdsoa_nssm}}. Figs. {\ref{fig:comp_ERs}} and {\ref{fig:comp_Qs}}, respectively, show current injection vs. ERs and Q-factors for both QDSOA AND gate and the proposed PC-QDSOA AND gate. To obtain ER of approximately 7.5 dB, the QDSOA AND gate requires 3000-mA current injection, whereas the proposed design requires only 5 mA. Likewise, Fig. {\ref{fig:comp_Qs}} shows that to obtain a Q-factor of approximately 5, the QDSOA AND gate requires 2800-mA current injection, whereas the proposed design requires only 4 mA. Moreover, the QDSOA is 2-mm long, whereas the proposed PC-QDSOA is 125-${\rm{\mu m}}$ long {\cite{qdsoa_nssm,pcqdsoa}}. This comparison demonstrates that the proposed gate reduces energy consumption and device volume compared with QDSOA AND gates.

\section{Conclusion}
We have investigated a PC-QDSOA all-optical AND gate that can operate at 160 Gb/s. The performance is evaluated by eye diagram, ER, and Q-factor. The simulation results show that the proposed gate can operate as an AND gate with ER of 8.58-dB and Q-factor of 7.41 at 160 Gb/s when current injection $I \geq$ 6 mA. This performance can be improved by increasing current injection, as this would decrease pattern effects. Moreover, by comparing the proposed gate with the QDSOA AND gate, it is found that the proposed gate significantly reduces energy consumption and device volume. The current injection required for the proposed gate is one out of six hundred of which required for the QDSOA AND gate.
\vskip3pt
\ack{This work was supported in part by JSPS KAKENHI Grant Numbers JP16K18108, JP17K06443.}

\vskip5pt

\noindent T. Matsumoto, K. Komatsu, G. Hosoya and H. Yashima (\textit{Tokyo University of Science, 6-3-1, Niijuku, Katsushika-ku, Tokyo, 1258585, Japan})
\vskip3pt

\noindent E-mail: t.m515621@gmail.com

\begin{thebibliography}{9}
\bibitem{3r_regeneration}
B. Sartorius, ``3R regeneration for all-optical networks, " {\it ICTON}, pp. 333--337, Aug. 2002.

\bibitem{qdsoa_nssm} 
K. Abedi and H. Taleb, ``Phase Recovery Acceleration in Quantum-Dot Semiconductor Optical Amplifiers, " {\it J. Lightw. Technol.}, vol. 32, no.12, pp. 237--241, Jun. 2012.

\bibitem{current_cpu}
(2017, Dec. 8). {\it Intel Core i7-5557U specifications} [Online]. Available: http://www.cpu-world.com/CPUs/Core\_i7/Intel-Core\%20i7-5557U\%20Mobile\%20processor.html

\bibitem{theory_of_qdsoa}
M. Sugawara, H. Ebe, N. Hatori, M. Ishida, Y. Arakawa, T. Akiyama, K. Otsubo and Y. Nakata, ``Theory of optical signal amplification and processing by quantum-dot semiconductor optical amplifiers, '' {\it PhysRevB}, vol. 69, no. 23, Jun. 2004.

\bibitem{pcqdsoa} 
H. Taleb and K. Abedi, ``Optical Gain, Phase, and Refractive Index Dynamics in Photonic Crystal Quantum-Dot Semiconductor Optical Amplifiers, " {\it IEEE J. Quantum Electron.}, vol. 50, no. 8, pp. 605--612, Aug. 2014.

\bibitem{mzi_gate}
D. Gayen and T. Chattopadhyay, ``Designing of Optimized All-Optical Half Adder Circuit Using Single Quantum-Dot Semiconductor Optical Amplifier Assisted Mach-Zehnder Interferometer, " {\it J. Lightw. Technol.}, vol. 31, no.12, pp. 2029--2035, Jun. 2013.

\end{thebibliography}

\end{document}
